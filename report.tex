\documentclass[10pt]{report}
\title{Software Engineering Large Practical - Audience Response System}
\author{Cameron Gray - s1230461}

\usepackage[margin=1.5cm]{geometry}
\usepackage{graphicx}
\usepackage{listings}

\begin{document}
	\maketitle
	
	\chapter*{Project Idea and Reasoning}
	\section*{Introduction}
	\paragraph{}
	For this project I built an "Audience Response System", similar to the "clicker" system used within
	the university.  The idea came about after seeing clear frustration with the current system that is in
	in use due to a few main failings - It is slow, the software is confusing to use and students are
	required to collect and look after a piece of hardware.  This would be implemented as an entirely web
	based application where both the tutor and students interact with the software using a web browser.
	
	\section*{Technologies}
	\paragraph{Backend}	
	I decided to use the Django web framework.  I chose this for several reasons - I am
	familiar with the Python programming language (which Django uses), I am already very well versed in
	Model, View, Controller web development and the framework is very well established and supported.
	
	\paragraph{CSS}
	I have used Twitter's Bootstrap framework to provide most of the CSS styling for this application -
	This allowed me to not only build a good looking application, it saved the time that I would otherwise
	have had to spend building a custom CSS template.  Bootstrap is also extremely well documented which
	will aid maintenance in the future since there is already very comprehensive documentation for the
	frontend CSS.
	
	\paragraph{Frontend Javascript}
	I have used the jQuery Javascript framework as it is an extremely well established framework that
	provides several time saving functions that would otherwise need implemented manually such as DOM
	manipulation and AJAX.
	
	\paragraph{Database}
	Throughout development I have used an SQLite database, this was used due to its simplistic nature and
	because it does not require that a database server is running on every machine that I am developing
	on.  In production use SQLite would not be suitable as it is not designed to support several
	concurrent users, in this case the application would need to use a more powerful database server such
	as PostgreSQL.  Thankfully due to the high level of abstraction from the database that Django provide,
	changing to a new database server is as simple as changing a single configuration file to reflect the
	settings for the new database - The underlying application would not require any changes.
	
	\section*{Improvements over current "Clicker" system}
	\paragraph{}
	This new web based audience response system has several clear advantages over the hardware based
	"Clicker" system that the University currently uses.
	
	\paragraph{No requirement on hardware}
	From the University's perspective, they would no longer need to loan out "Clickers" to students and
	deal with handling returns and losses.  There would also be no requirement for the specialised
	receiving hardware that is installed in lecture theatres.  This poses clear cost savings in terms of
	purchasing the system along with the cost of ongoing maintenance.  This would also benefit other users
	who may not be able to afford the specialised hardware required.
	
	\paragraph{System can be used over the Internet}
	Due to the current system's reliance on radio hardware, students can only respond from within the same
	room as the receiving hardware is located. Having a web based system means that students can respond
	to questions from anywhere with an internet connection which opens the system to be used for live web
	tutorials and online conferences.
	
	\section*{Disadvantage over the current "Clicker" system}
	\paragraph{Requires students to own hardware}
	Students are required to be able to provide their own hardware to respond to questions. In most cases
	this would not be an issue as the vast majority of students own at least one device that offers a web
	browser.
	
	\section*{Terminology}
	\paragraph{}
	Before proceeding I feel that it is important to clarify some of the system specific terminology that
	is used in both this report and in the software itself in the context of the system being used live
	in a lecture environment.
	
	\paragraph{Tutor}
	A tutor is a person who would be inputting questions, sending them out to students and looking at the
	responses. This person would most likely be the lecturer.  Every tutor has exactly one django user for
	authentication.
	
	\paragraph{Course}
	A course is a specific subject that is being taught, this would generally be directly related to the
	courses that the university teaches.  A course for example could be "Software Engineering Large
	Practical" or "Operating Systems".
	
	\paragraph{Session}
	A session is a group of questions, these would usually be grouped together in the same way that a
	lecture would - In most cases you would have one session for each lecture.  An example of a session
	could be "Git - Source Code Control" or "Memory Management".
	
	\paragraph{Session Run}
	A session run can be thought of as an instance of a session.  Each individual lecture would relate to
	a single session run.  The reason that they are differentiated is because the same lecture may be
	repeated at different times and the results from the questions should be kept separate between both of
	them.  A session run has both a time at which it was started and the single session that it is an
	instance of.
	
	\paragraph{Question Option}
	A question option would relate to a single possible answer to a multiple choice question.  It contains
	a "body" and is related to a single question. A question option can either be correct or incorrect. As
	an example, for the question "What colour is the sky?" the question options could be: Red (Inorrect),
	Green (Incorrect), Blue (Correct).
	
	\chapter*{Implementation Details}
	\section*{Database Design}
	\paragraph{}
	In order to improve understandability of the database, I created a diagram to represent the database
	using WWW SQL Designer\footnote{https://code.google.com/p/wwwsqldesigner/}. This tool allows the
	diagram to be exported and saved as an XML file meaning that the diagram can even be stored in version
	control (as I have done in "database.wwwsqldesigner").  The finalised diagram for the database that
	I used is as follows:
	
	\vspace{+10px}
	\includegraphics[width=\linewidth]{"report assets/database".png}
	
	\paragraph{}
	Using this diagram has proven extremely useful as it not only makes development easier, it aids
	maintainability down the line as developers can easily visualise exactly how the data is structured
	instead of manually having to look at the models or database itself.
	
	\section*{Code Structure}
	\subsection*{Backend}
	\paragraph{}
	Due to the use of the Django framework, the structure of code is very rigidly defined which means that
	it would be familiar to any Django developer who is required to work on the project.  The code is
	divided up into three different Django "apps".  The first powers the tutor area, the second powers the
	student area and the third app called "main" contains components that are shared by both the tutor and
	student areas.
	
	\paragraph{Models}
	This "main" app contains all of the models, the reasoning behind this is that in this system, all of
	the models are shared very closely between the two apps.  If I were	storing data that would be used
	only in the student area or only in the tutor area, then the models would be stored in the respective
	app rather than in "main".  The individual models themselves are kept very simple and mostly just
	define the data that is to be stored in them as well as default values.  Every model also contains a
	"\_\_str\_\_" method which is used to give each model a human readable string.  This is not used in
	the running of the application but makes the data a lot easier to understand when manually querying
	the models from the django shell or when using Django's premade admin area.
	
	\paragraph{Views}
	The views.py files in both the tutor and student apps holds most of the backed code. Django
	confusingly calls the file views.py but in the sense of MVC this file acts as a "controller" and the
	Django "templates" act like a view.  The main job of the code located in these view files is to take
	data recieved in the request and then either use this to insert data into the database, modify
	existing data in the database or to retrieve data from the database.  The result is then rendered into
	a template that is displayed to the user.
	
	\paragraph{Templates}
	The templates are stored both in the root directory (for base templates that area later extended) and
	in the individual apps themselves (for app specific templates).  Inheritance is used throughout the
	templates to ensure that code is not duplicated between pages, this means that if changes are to be
	made to a common area (e.g. the navigation bar), only one file needs to be changed.  This also
	prevents inconsistencies between pages.
	
	\paragraph{Project Settings}
	The flash\_response directory is a Django default directory which holds project level files including
	the main settings file and the base URL routes file.
	
	\paragraph{Decorators}
	The decorators.py file inside the main app is where I define Python decorator functions, at the moment
	there is a single decorator called "tutor\_course\_is\_selected" - When a function is decorated with
	this, the function can only execute if the tutor has selected a course from the course selection menu.
	If a course is not selected, the page will instead redirect to the tutor index page where a message
	will instruct the user to select a course.
	
	\paragraph{Context Processors}
	The context\_processors.py file in the tutor app contains a single context processor called 
	"course\_assignments\_processor."  This is used to ensure that the list of courses that the tutor
	is assigned to teach is always available in all templates in the tutor area since this list is used
	to create the common navigation bar.  This saves a lot of code reuse as otherwise, every single view
	that renders a template would have to get the courses and pass it into the template renderer.
	
	\subsection*{Frontend CSS}
	\paragraph{}
	This project uses the Twitter Bootstrap CSS framework therefore there is only a small amount of
	hand-written CSS.  All of the CSS is stored in the /assets/css/ directory.  The bootstrap* files are
	all part of the Bootstrap framework.  The style.css file was written by myself in order to add some
	extra styling and to override some of Bootstrap's default styles.
	
	\subsection*{Frontend Javascript}
	\paragraph{}
	All of my frontend Javascript is located in the /assets/js/ directory.  The bootstrap and jChart files
	are both libraries that I did not write but are used inside the application itself.   The student.js
	file provides the Javascript functionality for the student area and the tutor.js file provides the
	functionality for the tutor area.
	
	\paragraph{}
	This Javascript code has been written in an object oriented manner.  This is mostly to ensure that
	the individual parts of the Javascript code is kept separate from one another and so that the
	different components are clearly separated.
	
	\paragraph{}
	The frontend Javascript code in this application provides two main functions at the frontend.  I
	extensively	use AJAX in order to asynchronously transfer data between the backend server and the
	frontend software.  An example of this is the AJAX request that is made to the backend when the user
	picks an answer to a question.  The Javascript is also used to provide a lot of the user interface; it
	is used	for showing/hiding parts of the page as well as being used to populate parts of the page with
    dynamic data received from the backend.
    
    \paragraph{}
    I have relied heavily on the jQuery Javascript Library throughout this application as it provides
    extremely useful time-saving methods for AJAX and DOM manipulation.  It is also an extremely mature
    library that is used on the vast majority of modern websites and web applications.
    
    \section*{Testing}
    \paragraph{}
    For this project I chose to implement functional tests, due to how well they tie into a web
    application.  I have implemented them using Django's own testing framework and used their Test Client
    which allows you to send simulated POST and GET requests to the server and make assertions on the
    server's response.  These tests are located in /main/tests/.  For these tests I test both basic
    functionality such as adding a question and then checking it exists in the database as well as much
    more advanced functionality such as attempting to edit sessions in a course that the tutor is not
    assigned to (this should be blocked) and testing a full simulation of a question where 1000 random
    responses are sent to a question and the results recorded in the database are checked to ensure they
    match up with what was sent.  To run the tests you need to run:
    \begin{lstlisting}[language=bash]
    	$ python manage.py test -v 2
    \end{lstlisting}
    
	\section*{Interesting Points about the Implementation}
	\subsection*{Keeping everything in time}
	\paragraph{}
	Because the student area poll the backend every 2 seconds to check if a question is available, it
	becomes tricky to ensure that all students are presented with the question at practically the
	same time.  I was able to solve this problem by implementing the 5 second countdown that is seen
	between clicking to start a question and the question actually being seen by students.
	
	\paragraph{}
	During this countdown, the student clients will poll the backend and be presented with both the
	question as well as the number of milliseconds until they are to present the question to the user.
	Once this period of time has elapsed, the client will display the question to the user.  This proved
	to be a very successful and simple way to keep everything in time.
	
	\paragraph{}
	I also made sure to never use the time on the client device, everything is timed to the time set on
	the server, client devices only deal with time intervals (numbers of milliseconds) - This prevents
	time discrepancies across devices from causing problems.
	
	\subsection*{Counting users that are currently waiting to respond}
	\paragraph{}
	Another challenge I discovered was to be able to display the number of users waiting to receive a
	question in the session console.  I was able to solve this issue by having each student device send
	a UUID (unique identifier) along with every request to check if a question is available.  The server
	then stores this UUID in the database along with the date/time the user was last seen.  The
	application	then regards a student as waiting to respond to a question as long as the server has
	received a request from that user's UUID within the last 5 seconds.
    
    \chapter*{Instructions}
    \section*{Installation}
    \paragraph{}
    The following instructions detail how to install and run this project on DICE.
    
    \paragraph{Create Virtual Environment}
    Change to the directory that contains the application directory.  You will need to create a Python 3.4
    virtual environment.  To do this, run the following command:
    \begin{lstlisting}[language=bash]
    	$ pyvenv-3.4 ENV
    \end{lstlisting}
    
    \paragraph{Enter Virtual Environment}
    Before running any other parts of this project, you must switch into the virtual environment that you
    just created.  To do this run:
    \begin{lstlisting}[language=bash]
    	$ source ENV/bin/activate
    \end{lstlisting}
    To deactivate the virtual environment you can then run:
    \begin{lstlisting}[language=bash]
    	$ deactivate
    \end{lstlisting}
    
    \paragraph{Installing Dependences}
    This project has some dependencies that need to be installed.  These are stored in the 
    requirements.txt file and can be installed using pip as follows.  Remember, you must be in the virtual
    environment.  Change into the directory containing requirements.txt (The root of the project) and run:
    \begin{lstlisting}[language=bash]
    	$ pip install -r requirements.txt
    \end{lstlisting}
    
    \paragraph{Insert Sample Data}
    In order to insert the data you must run two commands, one to create the database itself from the
    models and the second to import the sample data which is sufficient for testing the system. Change to
    the flash\_response directory (the one containing manage.py) and run:
    \begin{lstlisting}[language=bash]
    	$ python manage.py syncdb
    	$ python manage.py loaddata ../sample\_data.json
    \end{lstlisting}
    \paragraph{}
    Running syncdb will ask you if you want to create a superuser, you can safely say "no" to this as it
    is not required in this case.
    
    \paragraph{Running Tests}
    The test suite can be executed by changing into the flash\_response directory (which contains the main
    code) and running:
    \begin{lstlisting}[language=bash]
    	$ python manage.py test -v 2
    \end{lstlisting}
    
    \paragraph{Starting the Server}
    In order to start using the software you must start the Django development server.  To do this, change
    into the flash\_response directory and execute:
    \begin{lstlisting}[language=bash]
    	$ python manage.py runserver
    \end{lstlisting}
    The server will now begin to listen on localhost at port 8000.  You can now access the software at 
    http://127.0.0.1:8000/
    
    \section*{Usage}
    \paragraph{}
    This section will give a rough tour of the application to allow you to gain enough understanding in
    order to be able to use it effectively.  Before doing any of these steps, ensure that the server is
    running as detailed above.
    
    \paragraph*{Logging In as a Tutor}
    To login as a tutor go to http://127.0.0.1:8000/.  You will be presented with a simple welcome page
    and a button to take you into the tutor area, click on this button to be presented with a login form.
    In the sample data there are three tutors named tutor1, tutor2 and tutor3, all with the password 1234. 
    Tutor1 is enrolled to teach Maths, tutor2 is enrolled to teach Computing and tutor3 is enrolled to
    teach both.  Pick one of these tutors and login.
    
    \paragraph*{Selecting a Course}
    When you log in you will be asked to select a course to manage, to do this use the drop down menu in
    the top right of the page to select a course, this will only show the courses that the tutor you have
    logged in as is enrolled to teach.
    
    \paragraph*{Sessions}
	The "Sessions" button at the top of the page will take you to the sessions page.  Here you can create
	new sessions, edit existing ones (such as adding/changing questions) or start a session to send
	questions to students.
	
	\paragraph*{Starting a Session}
	To run a session and therefore ask students questions, pick one from the sessions page and click
	"Start New Session".  This will open the session console.  In here you will see a blue box containing
	the "Response URL" - This is the URL that students would navigate to in order to answer questions. For
	testing I would recommend opening a few copies of this URL in different tabs/browser windows.  Notice
	how the session console will update to show the number of students currently waiting to respond.
	
	\paragraph*{Asking and responding to a question}
	To ask a question to the class, pick one from the list in the "Run a question" section and enter a
	time in seconds to represent how long the question should be run for.  Then press "Start Question."
	After a 5 second countdown, the question will start and the session console will display a countdown
	as well as live updating to display the number of responses received.  While the question is running,
	look at the pages you have opened with the response URL (they must be open before you start the
	question).  On these pages you should see the question along with the possible options, on each page,
	click one of the options to simulate a student answering  a question.  Once the question finishes, the
	session console will display a bar chart that represents the results of the question.
	
	\paragraph*{Viewing Reports}
	The reports section (accessed via the "Reports" link in the tutor navigation bar) allows you to
	generate reports for each session run.  This will rank questions by how accurately they were answered
	and by how many people responded to them.  On the Reports page, select a session and then select the
	date and time of the run from the "Session Run" box.  Now click "Generate Report" to display the
	report for that session run.
    
    \chapter*{Future Enhancements}
    \paragraph{}
    There are many different improvements/enhancements that I would implement in this application given
    more time.  These are detailed in this section.
    
    \section*{WebSockets}
    \paragraph{}
    At the moment the frontend Javascript communicates with the backend server entirely using AJAX
    requests.  This means that when the frontend is waiting for something it has to poll the backend at a
    regular interval until it receives the response that it is waiting for. This method is used in areas
    of the application such as when the student area is waiting for the tutor to start a question or when
    the tutor area is retrieving how many students have responded to a question while the question is
    running.
    
    \paragraph{}
    The problem with this method is that it creates a lot of unnecessary requests to the server.  For a
    prototype this is not a problem however if we were to scale this application up to hundreds of users
    it could become a problem.  For example, the student area polls to check if a question is available
    every two seconds.  Therefore if we had 100 students in a class, the server would be receiving
    $(100/2)*60=3000$ server calls a minute!  Now imagine if this was deployed across a university where
    there could be several lectures of 100 students using the system at any one time.  This could also be
    an issue if people are responding on battery operated devices as these additional server calls could
    potentially impact on battery life.
    
    \paragraph{}
    The solution to this problem would be to implement WebSockets.  Therefore instead of the frontend
    constantly polling the server for data, the frontend opens a socket connection to the server.  The
    frontend can then pass data down this to the server (as it would do with AJAX) but it has the
    advantage that the server can pass up data to the frontend.  Therefore instead of students' devices 
    constantly polling the server to see if there is a question available, the server will instead push
    the question out to the students.
    
    \paragraph{}
    Due to time restrictions this was not implemented in this version of the software however if the
    software were ever to be released or used in a live environment, this functionality would be
    implemented as a top priority.
    
    \section*{Change question while in progress}
    \paragraph{}
    Ideally I would like to make it possible for the tutor to make changes while the question is in
    progress, this would mean that the tutor could stop the question early or add more time if students
    are taking longer than expected to complete the questions.
    
    \section*{API}
    \paragraph{}
    I would like to create an API for the application, mostly for the student area.  This would allow
    applications for smartphones to be developed instead of requiring students to use a web browser.
    
    \section*{More Testing}
    \paragraph{}
    At the moment I have written a reasonable number of functional tests for the application.  However I
    would like to not only write more tests, I would like to run mutation testing on the application to
    make sure that the tests that I do have are comprehensive.
    
    \paragraph{}
    The application also turned out to be a lot more frontend-heavy than I originally predicted. Therefore
    I would also like to have some tests for this frontend.  I would do this using some sort of browser
    automation testing framework such as Selenium.
\end{document}