\documentclass[10pt]{report}
\title{Software Engineering Large Practical - Audience Response System}
\author{Cameron Gray - s1230461}

\usepackage[margin=1.5cm]{geometry}
\usepackage{graphicx}

\begin{document}
	\maketitle
	
	\section*{Introduction}
	\paragraph{}
	For this project I built an "Audience Response System", similar to the "clicker" system used within
	the university.  The idea came about after seeing clear frustration with the current system that is in
	in use due to a few main failings - It is slow, the software is confusing to use and students are
	required to collect and look after a piece of hardware.  This would be implemented as an entirely web
	based application where both the tutor and students interact with the software using a web browser.
	
	\section*{Technologies}
	\paragraph{Backend}	
	I decided to use the Django web framework.  I chose this for several reasons - I am
	familiar with the Python programming language (which Django uses), I am already very well versed in
	Model, View, Controller web development and the framework is very well established and supported.
	
	\paragraph{CSS}
	I have used Twitter's Bootstrap framework to provide most of the CSS styling for this application -
	This allowed me to not only build a good looking application, it saved the time that I would otherwise
	have had to spend building a custom CSS template.  Bootstrap is also extremely well documented which
	will aid maintenance in the future since there is already very comprehensive documentation for the
	frontend CSS.
	
	\paragraph{Frontend Javascript}
	I have used the jQuery Javascript framework as it is an extremely well established framework that
	provides several time saving functions that would otherwise need implemented manually such as DOM
	manipulation and AJAX.
	
	\paragraph{Database}
	Throughout development I have used an SQLite database, this was used due to its simplistic nature and
	because it does not require that a database server is running on every machine that I am developing
	on.  In production use SQLite would not be suitable as it is not designed to support several
	concurrent users, in this case the application would need to use a more powerful database server such
	as PostgreSQL.  Thankfully due to the high level of abstraction from the database that Django provide,
	changing to a new database server is as simple as changing a single configuration file to reflect the
	settings for the new database - The underlying application would not require any changes.
	
	
	
	\iffalse
	  I counteracted these issues by making an easy
	to use web application where students can respond to questions using hardware they already have such
	as a smartphone, a tablet or a laptop.  This also has the added advantage of allowing students to
	respond to questions from anywhere in the world, they do not have to be in the same room as 
	\fi
\end{document}